\documentclass{article}


% if you need to pass options to natbib, use, e.g.:
%     \PassOptionsToPackage{numbers, compress}{natbib}
% before loading format


% ready for submission
\usepackage{format}


% to compile a preprint version, e.g., for submission to arXiv, add add the
% [preprint] option:
%     \usepackage[preprint]{format}


% to compile a camera-ready version, add the [final] option, e.g.:
%     \usepackage[final]{format}


% to avoid loading the natbib package, add option nonatbib:
%    \usepackage[nonatbib]{format}

% \usepackage[dvipsnames]{xcolor}


\usepackage[utf8]{inputenc} % allow utf-8 input
\usepackage[T1]{fontenc}    % use 8-bit T1 fonts
\usepackage{hyperref}       % hyperlinks
\usepackage{url}            % simple URL typesetting
\usepackage{booktabs}       % professional-quality tables
\usepackage{amsfonts}       % blackboard math symbols
\usepackage{nicefrac}       % compact symbols for 1/2, etc.
\usepackage{microtype}      % microtypography
\usepackage[usenames,dvipsnames]{xcolor}         % colors

\usepackage{graphicx}
\usepackage{multirow}
\usepackage{subfigure}
\usepackage{a4wide}
\usepackage{fancyhdr}
\usepackage[Algoritmo]{algorithm}
\usepackage{algorithmic}
\usepackage{tikz}
\usepackage{empheq}
\usetikzlibrary{trees}
\usepackage{multirow}
\usepackage{amssymb,amsmath}
\usepackage{amsthm}
\usepackage{float}


\newcommand{\cmt}[1]{{\color{blue}{#1}}}


\title{TITLE}


% The \author macro works with any number of authors. There are two commands
% used to separate the names and addresses of multiple authors: \And and \AND.
%
% Using \And between authors leaves it to LaTeX to determine where to break the
% lines. Using \AND forces a line break at that point. So, if LaTeX puts 3 of 4
% authors names on the first line, and the last on the second line, try using
% \AND instead of \And before the third author name.


\author{%
  Anisio Lacerda%\thanks{Use footnote for providing further information
    %about author (webpage, alternative address)---\emph{not} for acknowledging
   % funding agencies.}
   \\
  Department of Computer Science\\
  Federal University of Minas Gerais\\
  Belo Horizonte, MG 30170-901 \\
  \texttt{anisio@dcc.ufmg.br} \\
  % examples of more authors
  % \And
  % Coauthor \\
  % Affiliation \\
  % Address \\
  % \texttt{email} \\
  % \AND
  % Coauthor \\
  % Affiliation \\
  % Address \\
  % \texttt{email} \\
  % \And
  % Coauthor \\
  % Affiliation \\
  % Address \\
  % \texttt{email} \\
  % \And
  % Coauthor \\
  % Affiliation \\
  % Address \\
  % \texttt{email} \\
}


\begin{document}


\maketitle


\begin{abstract}
  
\end{abstract}


\section{Introduction}

\cmt{\subsection{What is the broad question asked in my field?}
    \begin{itemize}
    \item article1 (i) one-sentence broad question + (ii) one-sentence research question
    \end{itemize}
    }
    
    \cmt{\subsection{Why is that question important?}
    \begin{itemize}
        \item article1 one-sentence research motivation
    \end{itemize}
    
    \paragraph{Find:}
    \begin{itemize}
        \item What common ground can you find among these diverse research questions?
        \item What broader research questions can you write that capture the spirit of multiple of these referenced articles?
        \item Write three research questions that cover a majority of the referenced articles:
        \begin{enumerate}
            \item \textbf{rq1:}
            \item \textbf{rq2:}
            \item \textbf{rq3:}
        \end{enumerate}
    \end{itemize}
    
    \paragraph{Research Gap:}\textit{Combine the weaknesses in Literature Synopsis into the research gap.}
    \begin{itemize}
        \item How do the weaknesses synergize?
        \item By adding all of those weaknesses together, what larger limitation is revealed?
        \item \emph{That larger limitation is your research gap!}
    \end{itemize}
    \begin{itemize}
        \item \textbf{GAP:} \textit{Currently, there is ...}
        \item \textbf{What does the literature currently do well?}
        \item \textbf{Despite those advances, what knowledge does the literature lack?}
        \item \textbf{What is the literature missing that makes it unable to fill that knowledge gap?}
    \end{itemize}
    
    \paragraph{Research Question:}\textcolor{OliveGreen}{\bf How ...?}
    \begin{itemize}
        \item The above research gap allows you to now define your research question.
        \item The research question is an inquiry whose answer will fill the research gap.
        \item \textbf{Answer:} How does the research question fill the research gap?
        \begin{itemize}
            \item ...
        \end{itemize}
        \item \textbf{Research Question Importance:}\textit{It is fine to share your peers' motivations. But it's nice to have your own little twist if possible.}
        \begin{itemize}
            \item \textbf{Answer:} How does this research improve society/literature knowledge/?
            \item \textbf{Answer:} What niche research topics might this research open up?
            \item \textbf{Answer:} What interdisciplinary connections would it forge?
            \item \textbf{Answer:} How might it impact technology or policy?
        \end{itemize}
    \end{itemize}
}

\paragraph{Related work and its relation to the current.}

\paragraph{Structure and contributions}

\section{Preliminaries and background}

\paragraph{Vision Transformers.}

\section{Methodologies}

\section{Experiments}

Our experiments are based on several state-of-the-art vision Transformers, namely ...

\cmt{\section*{Results}
    \subsection*{What main findings does the literature show to answer my field's research questions?}
    \begin{itemize}
        \item anchor article1 one-sentence main finding (include data, tables, and figures if helpful).
    \end{itemize}
    
    \begin{itemize}
        \item related article1 one-sentence main finding
    \end{itemize}
}

\cmt{\section*{Relevant experimental setups}
\begin{itemize}
    \item \textbf{Datasets.}
    \begin{itemize}
        \item article1
    \end{itemize}
    \item \textbf{Backbones.} 
    \begin{itemize}
        \item article1
    \end{itemize}
    \item \textbf{Implementation details.}
    \begin{itemize}
        \item article1
    \end{itemize}
\end{itemize}
}


\subsection{Dataset1}

\paragraph{Experimental setup.}

\paragraph{Results}

\subsection{Dataset2}

\paragraph{Experimental setup.}

\paragraph{Results}

\section{Discussion}

\paragraph{Aspect1}

\paragraph{Conclusion}

\bibliographystyle{plain}
\bibliography{refs}



%%%%%%%%%%%%%%%%%%%%%%%%%%%%%%%%%%%%%%%%%%%%%%%%%%%%%%%%%%%%
\section*{Checklist}


%%% BEGIN INSTRUCTIONS %%%
The checklist follows the references.  Please
read the checklist guidelines carefully for information on how to answer these
questions.  For each question, change the default \answerTODO{} to \answerYes{},
\answerNo{}, or \answerNA{}.  You are strongly encouraged to include a {\bf
justification to your answer}, either by referencing the appropriate section of
your paper or providing a brief inline description.  For example:
\begin{itemize}
  \item Did you include the license to the code and datasets? \answerYes{See Section~\ref{gen_inst}.}
  \item Did you include the license to the code and datasets? \answerNo{The code and the data are proprietary.}
  \item Did you include the license to the code and datasets? \answerNA{}
\end{itemize}
Please do not modify the questions and only use the provided macros for your
answers.  Note that the Checklist section does not count towards the page
limit.  In your paper, please delete this instructions block and only keep the
Checklist section heading above along with the questions/answers below.
%%% END INSTRUCTIONS %%%


\begin{enumerate}


\item For all authors...
\begin{enumerate}
  \item Do the main claims made in the abstract and introduction accurately reflect the paper's contributions and scope?
    \answerTODO{}
  \item Did you describe the limitations of your work?
    \answerTODO{}
  \item Did you discuss any potential negative societal impacts of your work?
    \answerTODO{}
  \item Have you read the ethics review guidelines and ensured that your paper conforms to them?
    \answerTODO{}
\end{enumerate}


\item If you are including theoretical results...
\begin{enumerate}
  \item Did you state the full set of assumptions of all theoretical results?
    \answerTODO{}
        \item Did you include complete proofs of all theoretical results?
    \answerTODO{}
\end{enumerate}


\item If you ran experiments...
\begin{enumerate}
  \item Did you include the code, data, and instructions needed to reproduce the main experimental results (either in the supplemental material or as a URL)?
    \answerTODO{}
  \item Did you specify all the training details (e.g., data splits, hyperparameters, how they were chosen)?
    \answerTODO{}
        \item Did you report error bars (e.g., with respect to the random seed after running experiments multiple times)?
    \answerTODO{}
        \item Did you include the total amount of compute and the type of resources used (e.g., type of GPUs, internal cluster, or cloud provider)?
    \answerTODO{}
\end{enumerate}


\item If you are using existing assets (e.g., code, data, models) or curating/releasing new assets...
\begin{enumerate}
  \item If your work uses existing assets, did you cite the creators?
    \answerTODO{}
  \item Did you mention the license of the assets?
    \answerTODO{}
  \item Did you include any new assets either in the supplemental material or as a URL?
    \answerTODO{}
  \item Did you discuss whether and how consent was obtained from people whose data you're using/curating?
    \answerTODO{}
  \item Did you discuss whether the data you are using/curating contains personally identifiable information or offensive content?
    \answerTODO{}
\end{enumerate}


\item If you used crowdsourcing or conducted research with human subjects...
\begin{enumerate}
  \item Did you include the full text of instructions given to participants and screenshots, if applicable?
    \answerTODO{}
  \item Did you describe any potential participant risks, with links to Institutional Review Board (IRB) approvals, if applicable?
    \answerTODO{}
  \item Did you include the estimated hourly wage paid to participants and the total amount spent on participant compensation?
    \answerTODO{}
\end{enumerate}


\end{enumerate}


%%%%%%%%%%%%%%%%%%%%%%%%%%%%%%%%%%%%%%%%%%%%%%%%%%%%%%%%%%%%


\appendix


\section{Appendix}


Optionally include extra information (complete proofs, additional experiments and plots) in the appendix.
This section will often be part of the supplemental material.


\end{document}