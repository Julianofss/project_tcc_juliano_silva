% --
\section{\color{BrickRed}\texttt{REF}, ARTICLENAME, \texttt{VENUEYEAR}~\cite{zhang@aaai22}}

\subsection{Research Question}

\cmt{
\begin{itemize}
    \item \textit{Look near the end of the Introduction: Write a one-sentence summary of the article's research question. If the article gives a hypothesis, write a one-sentence summary of that too.}
\end{itemize}
}

\subsection{Motivation}

\cmt{
\begin{itemize}
    \item \textit{Look near the beginning of the article\'s Introduction: Write a one-sentence summary of the social motivation. How will the article solve social, environmental, or other problems?}

    \item \textit{Look throughout the Introduction\'s Background/Lit Review portion: Write a one sentence summary of the research gap. How will the article fill a lack of knowledge in the literature?}
\end{itemize}
}

\subsection{Results}

\cmt{
\begin{itemize}
    \item \textit{Look in the Results \& Discussion, especially the figures, tables, and subsection headers: Write a one\-sentence summary of the main finding. What finding directly answers the research question?}
    
    \item \textit{Write a one\-sentence summary of each supplementary finding. What context do these supplementary findings add to the main finding?}
    
    \item \textit{Copy and paste any relevant figures and/or tables.}
\end{itemize}
}

\subsection{Experiments}

\cmt{
\begin{itemize}
    \item \textit{Look in the Experiments section: Write a one-paragraph summary of the experimental methods. Describe each phase of the experiment, how those phases interact. Draw/copy a flowchart for complex experimental setup.}
    
    \item \textit{How was this research accomplished?}
    
    \item \textit{This section explains:}
    \begin{itemize}
        \item \textit{Theory:}
        
        \item \textit{Experimental design:}
        
        \item \textit{Metrics:}
        
        \item \textit{Statistical analysis:}
    \end{itemize}
\end{itemize}
}

\subsection{Methodology}
\cmt{
    \begin{itemize}
        \item \textit{Describe what inputs they require, and what outputs they generate. Draw/copy a flowchart for complex methods. Write a one-paragraph of the main difference in the proposed method.}
    \end{itemize}
}

\begin{itemize}
    \item In/Out of the proposed method:
    \begin{itemize}
        \item What input data they require:
        \item What output data they generate:
        \item How was this research accomplished?
    \end{itemize}
    
    \item This section explains:
    \begin{itemize}
        \item Background:
        \item Theory:
    \end{itemize}
    
    \item Key difference:
\end{itemize}

\subsection{Impressions}

\begin{itemize}
    \item General impression:\cmt{
    \begin{itemize}
        \item 
    \end{itemize}
    }
    
    \item Final message:\cmt{
    \begin{itemize}
        \item 
    \end{itemize}
    }
    
    \item How can I adjust the article's research question to better fit its motivation?\cmt{
    \begin{itemize}
        \item 
    \end{itemize}
    }
    
    \item Does the article answer the research question fully fill the research gap identified?\cmt{
    \begin{itemize}
        \item 
    \end{itemize}
    }
    
    \item What are the limitations of the study? How could I reduce these limitations?\cmt{
    \begin{itemize}
        \item 
    \end{itemize}
    }
    
    \item What assumptions does the study make? Are those assumptions correct? If not, how must the study change if these assumptions are abandoned?\cmt{
    \begin{itemize}
        \item 
    \end{itemize}
    }
    
\end{itemize}