\documentclass[11pt,dvipsnames]{article}

\usepackage[dvipsnames]{xcolor}

\usepackage{bibentry}
\nobibliography*

%\usepackage[brazil]{babel}
%\usepackage[latin1]{inputenc}
\usepackage{inputenc}
\usepackage{graphicx}
\usepackage{multirow}
\usepackage{subfigure}
\usepackage{a4wide}
\usepackage{fancyhdr}
\usepackage[Algoritmo]{algorithm}
\usepackage{algorithmic}
\usepackage{tikz}
\usepackage{empheq}
\usetikzlibrary{trees}
\usepackage{multirow}
\usepackage{subfigure}
\usepackage{amssymb,amsmath}
\usepackage{amsthm,amsfonts}
\usepackage{float}


\newcommand{\cmt}[1]{{\color{blue}{#1}}}

\newcommand{\myref}[1]{\bold{#1}{#1}}

\pagestyle{fancy}
\lhead{Literature Review}
\rhead{\thepage}
%\addtolength{\headheight}{\baselineskip}
\renewcommand{\headrulewidth}{0.1pt}
\renewcommand{\footrulewidth}{0.1pt}
\lfoot{NOME}
\rfoot{DATE}
\cfoot{}

\sloppy

\begin{document}
% \floatname{algorithm}{Algoritmo}
% \renewcommand{\algorithmicend}{\textbf{fim}}
% \renewcommand{\algorithmicif}{\textbf{se}}
% \renewcommand{\algorithmicthen}{\textbf{então}}
% \renewcommand{\algorithmicelse}{\textbf{senão}}
% \renewcommand{\algorithmicfor}{\textbf{para}}
% \renewcommand{\algorithmicdo}{\textbf{faça}}
% \renewcommand{\algorithmicwhile}{\textbf{enquanto}}
\thispagestyle{empty}

\begin{center}
\begin{minipage}[l]{10cm}{
\center
Trabalho de Conclus\~{a}o de Curso \\
}\end{minipage}
 \vfill
 \begin{minipage}[l]{11cm}{
  \begin{center}
  \Large{Literature Review}
  \end{center}
}\end{minipage}
\end{center}
 \vspace*{8cm}
 \begin{center}
 \begin{minipage}[l]{10cm}{
 \center NOME\\
 DATE\\
 }
 \end{minipage}
 \end{center}
%\newpage
%\thispagestyle{empty}
%\tableofcontents
\newpage

% -----

\thispagestyle{empty}
\tableofcontents
\newpage

\section{Instructions}



\subsection{The Goals of Literature Review:}\cmt{

\begin{itemize}
    \item In the literature review phase, we search the academic literature to give our project context and develop its research gap.
    \item It pursues interdependence by putting our project in the context of other researcher's work.
\end{itemize}

}

\subsection{Reasons to cite literature:}\cmt{

\begin{enumerate}
    \item to motivate our work and explain its value,
    \item to provide technical background information,
    \item to compare our findings against other articles',
    \item to communicate our project's scientific novelty by framing the research gap, and
    \item to justify our experimental methods.
\end{enumerate}

\begin{itemize}
    \item  The idea is that, whereas we read, we find gaps in scientific understanding.

    \item These gaps represent missing knowledge, and we identify them by posing questions that the literature cannot answer.

    \item These inquiries evolve into our project’s research question. 
    
    \item And by building our research question around these gaps and answering it via existing experimental methods, we feel confident that our project can likely solve that research question, produce scientific novelty, and intrigue the researchers in our field.
\end{itemize}
}

\subsection{Don't Read; Skim} \cmt{
\begin{itemize}
    \item We won't fully read most articles, at least not from start to finish~\footnote{The exception to this skimming is when we read our \textit{anchor papers}, i.e. those directly related to ours. For instance, our baseline methods.}.
    \item 
\end{itemize}
}

\subsection{Take home}\cmt{

\begin{itemize}
    \item From a practical standpoint, literature review reduces our overall effort. 
    
    \begin{itemize}
        \item It makes the writing process easier, 
        \item Reduces research mistakes, and 
        \item Helps us pursue a meaningful research question.
    \end{itemize}
    
    \item \textbf{Your goal is to answer a specific question in a narrow niche of your broader research field.}
\end{itemize}
}




\newpage


% --
% --
\section{\color{BrickRed}\texttt{REF}, ARTICLENAME, \texttt{VENUEYEAR}~\cite{zhang@aaai22}}

\subsection{Research Question}

\cmt{
\begin{itemize}
    \item \textit{Look near the end of the Introduction: Write a one-sentence summary of the article's research question. If the article gives a hypothesis, write a one-sentence summary of that too.}
\end{itemize}
}

\subsection{Motivation}

\cmt{
\begin{itemize}
    \item \textit{Look near the beginning of the article\'s Introduction: Write a one-sentence summary of the social motivation. How will the article solve social, environmental, or other problems?}

    \item \textit{Look throughout the Introduction\'s Background/Lit Review portion: Write a one sentence summary of the research gap. How will the article fill a lack of knowledge in the literature?}
\end{itemize}
}

\subsection{Results}

\cmt{
\begin{itemize}
    \item \textit{Look in the Results \& Discussion, especially the figures, tables, and subsection headers: Write a one\-sentence summary of the main finding. What finding directly answers the research question?}
    
    \item \textit{Write a one\-sentence summary of each supplementary finding. What context do these supplementary findings add to the main finding?}
    
    \item \textit{Copy and paste any relevant figures and/or tables.}
\end{itemize}
}

\subsection{Experiments}

\cmt{
\begin{itemize}
    \item \textit{Look in the Experiments section: Write a one-paragraph summary of the experimental methods. Describe each phase of the experiment, how those phases interact. Draw/copy a flowchart for complex experimental setup.}
    
    \item \textit{How was this research accomplished?}
    
    \item \textit{This section explains:}
    \begin{itemize}
        \item \textit{Theory:}
        
        \item \textit{Experimental design:}
        
        \item \textit{Metrics:}
        
        \item \textit{Statistical analysis:}
    \end{itemize}
\end{itemize}
}

\subsection{Methodology}
\cmt{
    \begin{itemize}
        \item \textit{Describe what inputs they require, and what outputs they generate. Draw/copy a flowchart for complex methods. Write a one-paragraph of the main difference in the proposed method.}
    \end{itemize}
}

\begin{itemize}
    \item In/Out of the proposed method:
    \begin{itemize}
        \item What input data they require:
        \item What output data they generate:
        \item How was this research accomplished?
    \end{itemize}
    
    \item This section explains:
    \begin{itemize}
        \item Background:
        \item Theory:
    \end{itemize}
    
    \item Key difference:
\end{itemize}

\subsection{Impressions}

\begin{itemize}
    \item General impression:\cmt{
    \begin{itemize}
        \item 
    \end{itemize}
    }
    
    \item Final message:\cmt{
    \begin{itemize}
        \item 
    \end{itemize}
    }
    
    \item How can I adjust the article's research question to better fit its motivation?\cmt{
    \begin{itemize}
        \item 
    \end{itemize}
    }
    
    \item Does the article answer the research question fully fill the research gap identified?\cmt{
    \begin{itemize}
        \item 
    \end{itemize}
    }
    
    \item What are the limitations of the study? How could I reduce these limitations?\cmt{
    \begin{itemize}
        \item 
    \end{itemize}
    }
    
    \item What assumptions does the study make? Are those assumptions correct? If not, how must the study change if these assumptions are abandoned?\cmt{
    \begin{itemize}
        \item 
    \end{itemize}
    }
    
\end{itemize}

% --
\section{\texttt{Khan, et.al.}, Transformers in Vision: A Survey, \texttt{ACM Surveys 2021}~\cite{khan@acmsurveys21}}

\subsection{Research Question}

\cmt{
\begin{itemize}
    \item \textit{Look near the end of the Introduction: Write a one-sentence summary of the article's research question. If the article gives a hypothesis, write a one-sentence summary of that too.}
\end{itemize}
}

\subsection{Motivation}

\cmt{
\begin{itemize}
    \item \textit{Look near the beginning of the article\'s Introduction: Write a one-sentence summary of the social motivation. How will the article solve social, environmental, or other problems?}

    \item \textit{Look throughout the Introduction\'s Background/Lit Review portion: Write a one sentence summary of the research gap. How will the article fill a lack of knowledge in the literature?}
\end{itemize}
}

\subsection{Results}

\cmt{
\begin{itemize}
    \item \textit{Look in the Results \& Discussion, especially the figures, tables, and subsection headers: Write a one\-sentence summary of the main finding. What finding directly answers the research question?}
    
    \item \textit{Write a one\-sentence summary of each supplementary finding. What context do these supplementary findings add to the main finding?}
    
    \item \textit{Copy and paste any relevant figures and/or tables.}
\end{itemize}
}

\subsection{Experiments}

\cmt{
\begin{itemize}
    \item \textit{Look in the Experiments section: Write a one-paragraph summary of the experimental methods. Describe each phase of the experiment, how those phases interact. Draw/copy a flowchart for complex experimental setup.}
    
    \item \textit{How was this research accomplished?}
    
    \item \textit{This section explains:}
    \begin{itemize}
        \item \textit{Theory:}
        
        \item \textit{Experimental design:}
        
        \item \textit{Metrics:}
        
        \item \textit{Statistical analysis:}
    \end{itemize}
\end{itemize}
}

\subsection{Methodology}
\cmt{
    \begin{itemize}
        \item \textit{Describe what inputs they require, and what outputs they generate. Draw/copy a flowchart for complex methods. Write a one-paragraph of the main difference in the proposed method.}
    \end{itemize}
}

\begin{itemize}
    \item In/Out of the proposed method:
    \begin{itemize}
        \item What input data they require:
        \item What output data they generate:
        \item How was this research accomplished?
    \end{itemize}
    
    \item This section explains:
    \begin{itemize}
        \item Background:
        \item Theory:
    \end{itemize}
    
    \item Key difference:
\end{itemize}

\subsection{Impressions}

\begin{itemize}
    \item General impression:\cmt{
    \begin{itemize}
        \item 
    \end{itemize}
    }
    
    \item Final message:\cmt{
    \begin{itemize}
        \item 
    \end{itemize}
    }
    
    \item How can I adjust the article's research question to better fit its motivation?\cmt{
    \begin{itemize}
        \item 
    \end{itemize}
    }
    
    \item Does the article answer the research question fully fill the research gap identified?\cmt{
    \begin{itemize}
        \item 
    \end{itemize}
    }
    
    \item What are the limitations of the study? How could I reduce these limitations?\cmt{
    \begin{itemize}
        \item 
    \end{itemize}
    }
    
    \item What assumptions does the study make? Are those assumptions correct? If not, how must the study change if these assumptions are abandoned?\cmt{
    \begin{itemize}
        \item 
    \end{itemize}
    }
    
\end{itemize}

% --
% --
\section{\texttt{Han et. al.}, A survey on vision transformer, \texttt{IEEE Transactions on Pattern Analysis and Machine Intelligence}~\cite{han@tpami22}}

\subsection{Research Question}

\cmt{
\begin{itemize}
    \item \textit{Look near the end of the Introduction: Write a one-sentence summary of the article's research question. If the article gives a hypothesis, write a one-sentence summary of that too.}
\end{itemize}
}

\subsection{Motivation}

\cmt{
\begin{itemize}
    \item \textit{Look near the beginning of the article\'s Introduction: Write a one-sentence summary of the social motivation. How will the article solve social, environmental, or other problems?}

    \item \textit{Look throughout the Introduction\'s Background/Lit Review portion: Write a one sentence summary of the research gap. How will the article fill a lack of knowledge in the literature?}
\end{itemize}
}

\subsection{Results}

\cmt{
\begin{itemize}
    \item \textit{Look in the Results \& Discussion, especially the figures, tables, and subsection headers: Write a one\-sentence summary of the main finding. What finding directly answers the research question?}
    
    \item \textit{Write a one\-sentence summary of each supplementary finding. What context do these supplementary findings add to the main finding?}
    
    \item \textit{Copy and paste any relevant figures and/or tables.}
\end{itemize}
}

\subsection{Experiments}

\cmt{
\begin{itemize}
    \item \textit{Look in the Experiments section: Write a one-paragraph summary of the experimental methods. Describe each phase of the experiment, how those phases interact. Draw/copy a flowchart for complex experimental setup.}
    
    \item \textit{How was this research accomplished?}
    
    \item \textit{This section explains:}
    \begin{itemize}
        \item \textit{Theory:}
        
        \item \textit{Experimental design:}
        
        \item \textit{Metrics:}
        
        \item \textit{Statistical analysis:}
    \end{itemize}
\end{itemize}
}

\subsection{Methodology}
\cmt{
    \begin{itemize}
        \item \textit{Describe what inputs they require, and what outputs they generate. Draw/copy a flowchart for complex methods. Write a one-paragraph of the main difference in the proposed method.}
    \end{itemize}
}

\begin{itemize}
    \item In/Out of the proposed method:
    \begin{itemize}
        \item What input data they require:
        \item What output data they generate:
        \item How was this research accomplished?
    \end{itemize}
    
    \item This section explains:
    \begin{itemize}
        \item Background:
        \item Theory:
    \end{itemize}
    
    \item Key difference:
\end{itemize}

\subsection{Impressions}

\begin{itemize}
    \item General impression:\cmt{
    \begin{itemize}
        \item 
    \end{itemize}
    }
    
    \item Final message:\cmt{
    \begin{itemize}
        \item 
    \end{itemize}
    }
    
    \item How can I adjust the article's research question to better fit its motivation?\cmt{
    \begin{itemize}
        \item 
    \end{itemize}
    }
    
    \item Does the article answer the research question fully fill the research gap identified?\cmt{
    \begin{itemize}
        \item 
    \end{itemize}
    }
    
    \item What are the limitations of the study? How could I reduce these limitations?\cmt{
    \begin{itemize}
        \item 
    \end{itemize}
    }
    
    \item What assumptions does the study make? Are those assumptions correct? If not, how must the study change if these assumptions are abandoned?\cmt{
    \begin{itemize}
        \item 
    \end{itemize}
    }
    
\end{itemize}



% --
\section{\texttt{Matsoukas, et.al.}, Is it time to replace CNNs with Transformers for medical images, \texttt{arXiv21}~\cite{matsoukas@arxiv21}}
\cmt{
OBS: submitted and rejected at ICLR22 as {\bf Can Transformers Replace CNNs for Medical Image Classification?} under the argument that it is a too broad study, which would be more fitted for a journal submission.
}

\subsection{Research Question}

\cmt{
\begin{itemize}
    \item \textit{Look near the end of the Introduction: Write a one-sentence summary of the article's research question. If the article gives a hypothesis, write a one-sentence summary of that too.}
\end{itemize}
}

\subsection{Motivation}

\cmt{
\begin{itemize}
    \item \textit{Look near the beginning of the article\'s Introduction: Write a one-sentence summary of the social motivation. How will the article solve social, environmental, or other problems?}

    \item \textit{Look throughout the Introduction\'s Background/Lit Review portion: Write a one sentence summary of the research gap. How will the article fill a lack of knowledge in the literature?}
\end{itemize}
}

\subsection{Results}

\cmt{
\begin{itemize}
    \item \textit{Look in the Results \& Discussion, especially the figures, tables, and subsection headers: Write a one\-sentence summary of the main finding. What finding directly answers the research question?}
    
    \item \textit{Write a one\-sentence summary of each supplementary finding. What context do these supplementary findings add to the main finding?}
    
    \item \textit{Copy and paste any relevant figures and/or tables.}
\end{itemize}
}

\subsection{Experiments}

\cmt{
\begin{itemize}
    \item \textit{Look in the Experiments section: Write a one-paragraph summary of the experimental methods. Describe each phase of the experiment, how those phases interact. Draw/copy a flowchart for complex experimental setup.}
    
    \item \textit{How was this research accomplished?}
    
    \item \textit{This section explains:}
    \begin{itemize}
        \item \textit{Theory:}
        
        \item \textit{Experimental design:}
        
        \item \textit{Metrics:}
        
        \item \textit{Statistical analysis:}
    \end{itemize}
\end{itemize}
}

\subsection{Methodology}
\cmt{
    \begin{itemize}
        \item \textit{Describe what inputs they require, and what outputs they generate. Draw/copy a flowchart for complex methods. Write a one-paragraph of the main difference in the proposed method.}
    \end{itemize}
}

\begin{itemize}
    \item In/Out of the proposed method:
    \begin{itemize}
        \item What input data they require:
        \item What output data they generate:
        \item How was this research accomplished?
    \end{itemize}
    
    \item This section explains:
    \begin{itemize}
        \item Background:
        \item Theory:
    \end{itemize}
    
    \item Key difference:
\end{itemize}

\subsection{Impressions}

\begin{itemize}
    \item General impression:\cmt{
    \begin{itemize}
        \item 
    \end{itemize}
    }
    
    \item Final message:\cmt{
    \begin{itemize}
        \item 
    \end{itemize}
    }
    
    \item How can I adjust the article's research question to better fit its motivation?\cmt{
    \begin{itemize}
        \item 
    \end{itemize}
    }
    
    \item Does the article answer the research question fully fill the research gap identified?\cmt{
    \begin{itemize}
        \item 
    \end{itemize}
    }
    
    \item What are the limitations of the study? How could I reduce these limitations?\cmt{
    \begin{itemize}
        \item 
    \end{itemize}
    }
    
    \item What assumptions does the study make? Are those assumptions correct? If not, how must the study change if these assumptions are abandoned?\cmt{
    \begin{itemize}
        \item 
    \end{itemize}
    }
    
\end{itemize}


\bibliographystyle{plain}
\bibliography{refs}

\end{document}